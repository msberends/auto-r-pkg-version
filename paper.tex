%% start with classes: 
%% - article: type of document
%% - shortnames: use 'et al.' after first author
%% - nojss: strip JSS-style completely, for preprinting at bioRxiv
\documentclass[article, shortnames]{jss}

%% -- LaTeX packages and custom commands ---------------------------------------

%% recommended packages
\usepackage{thumbpdf,lmodern}
\graphicspath{{Figures/}} % folder with figures

\usepackage[utf8]{inputenc}

%% another package (only for this demo article)
\usepackage{framed}

%% added package to allow multirow
\usepackage{multirow}

%% added package to rotate table 3 clockwise, see Appendix
\usepackage{rotating}

%% added package for pretty formula writing the matching score
\usepackage{amsmath}

% added package to support checkmarks in table 6
\usepackage{amssymb}

%% \setcitestyle{numbers}
%% way better readable with numbers, but not JSS style... Their example paper has it like this ("cite" and "citep"):
%% `as was earlier shown by \cite{Smith}.` = "as was earlier shown by Smith (2019)."
%% `as was earlier shown \citep{Smith}.`   = "as was earlier shown (Smith, 2019)."

%% new custom commands
\newcommand{\class}[1]{`\code{#1}'}
\newcommand{\fct}[1]{\code{#1()}}

%% For Sweave-based articles about R packages:
%% need no \usepackage{Sweave}


%% -- Article metainformation (author, title, ...) -----------------------------

%% - \title{} in title case
%% - \Plaintitle{} without LaTeX markup (if any)
%% - \Shorttitle{} with LaTeX markup (if any), used as running title
\title{\pkg{AMR} -- An \proglang{R} Package for Working with Antimicrobial Resistance Data}
\Plaintitle{AMR -- An R Package for Working with Antimicrobial Resistance Data}
\Shorttitle{\pkg{AMR} -- An \proglang{R} Package for Working with Antimicrobial Resistance Data}

%% - \author{} with primary affiliation
%% - \Plainauthor{} without affiliations
%% - Separate authors by \And or \AND (in \author) or by comma (in \Plainauthor).
%% - \AND starts a new line, \And does not.
\author{Matthijs S. Berends\\
          University of Groningen
   \And Christian F. Luz\\
          University of Groningen
   \AND Alexander W. Friedrich\\
          University of Groningen
   \And Bhanu N. M. Sinha\\
          University of Groningen
   \AND Casper J. Albers\\
          University of Groningen
   \And Corinna Glasner\\
          University of Groningen
}
\Plainauthor{Berends MS, Luz CF, Friedrich AW, Sinha BNM, Albers CJ, Glasner C}

%% - \Address{} of at least one author
%% - May contain multiple affiliations for each author
%%   (in extra lines, separated by \emph{and}\\).
%% - May contain multiple authors for the same affiliation
%%   (in the same first line, separated by comma).
\Address{
  Matthijs S. Berends\\
    Certe Medical Diagnostics and Advice Foundation\\
    Van Swietenlaan 2\\
    9728 NZ Groningen, The Netherlands\\
    \emph{and}\\
    University of Groningen\\
    University Medical Center Groningen\\
    Department of Medical Microbiology and Infection Prevention\\
    Hanzeplein 1\\
    9713 GZ Groningen, The Netherlands\\
    E-mail:~\email{m.berends@certe.nl}, \email{m.s.berends@umcg.nl}\\
    
  Christian F. Luz\\
    University of Groningen\\
    University Medical Center Groningen\\
    Department of Medical Microbiology and Infection Prevention\\
    Hanzeplein 1\\
    9713 GZ Groningen, The Netherlands\\
    
  Alexander W. Friedrich\\
    University of Groningen\\
    University Medical Center Groningen\\
    Department of Medical Microbiology and Infection Prevention\\
    Hanzeplein 1\\
    9713 GZ Groningen, The Netherlands\\
    
  Bhanu N.M. Sinha\\
    University of Groningen\\
    University Medical Center Groningen\\
    Department of Medical Microbiology and Infection Prevention\\
    Hanzeplein 1\\
    9713 GZ Groningen, The Netherlands\\
    
  Casper J. Albers\\
    University of Groningen\\
    Heymans Institute for Psychological Research\\
    Grote Kruisstraat 2/1\\
    9712 TS Groningen, The Netherlands\\
    
  Corinna Glasner\\
    University of Groningen\\
    University Medical Center Groningen\\
    Department of Medical Microbiology and Infection Prevention\\
    Hanzeplein 1\\
    9713 GZ Groningen, The Netherlands\\
}

%% - \Abstract{} almost as usual
\Abstract{
  % This short article illustrates how to write a manuscript for the
  % \emph{Journal of Statistical Software} (JSS) using its {\LaTeX} style files.
  % Generally, we ask to follow JSS's style guide and FAQs precisely. Also,
  % it is recommended to keep the {\LaTeX} code as simple as possible,
  % i.e., avoid inclusion of packages/commands that are not necessary.
  % For outlining the typical structure of a JSS article some brief text snippets
  % are employed that have been inspired by \cite{Zeileis+Kleiber+Jackman:2008},
  % discussing count data regression in \proglang{R}. Editorial comments and
  % instructions are marked by vertical bars.
%
  Antimicrobial resistance is an increasing threat to global health. 
Evidence for this trend is generated in microbiological laboratories through
testing microorganisms for resistance against antimicrobial agents. 
International standards and guidelines are in place for this process as well
as for reporting data on (inter-)national levels.  However, there is a gap
in the availability of standardized and reproducible tools for working with
laboratory data to produce the required reports.  It is known that extensive
efforts in data cleaning and validation are required when working with data
from laboratory information systems.  Furthermore, the global spread and
relevance of antimicrobial resistance demands to incorporate international
reference data in the analysis process.

  In this paper, we introduce the \pkg{AMR} package for \proglang{R} that
aims at closing this gap by providing tools to simplify antimicrobial
resistance data cleaning and analysis, while incorporating international
guidelines and scientifically reliable reference data.  The \pkg{AMR}
package enables standardized and reproducible antimicrobial resistance
analyses, including the application of evidence-based rules, determination
of first isolates, translation of various codes for microorganisms and
antimicrobial agents, determination of (multi-drug) resistant
microorganisms, and calculation of antimicrobial resistance, prevalence and
future trends.  The \pkg{AMR} package works independently of any laboratory
information system and provides several functions to integrate into
international workflows (e.g.,~\pkg{WHONET} software provided by the World Health
Organization).  }

%% - \Keywords{} with LaTeX markup, at least one required
%% - \Plainkeywords{} without LaTeX markup (if necessary)
%% - Should be comma-separated and in sentence case.
\Keywords{antimicrobial resistance, data analysis, \proglang{R}, software, epidemiology}
\Plainkeywords{antimicrobial resistance, data analysis, R, software, epidemiology}

\begin{document}


%% -- Introduction -------------------------------------------------------------

%% - In principle "as usual".
%% - But should typically have some discussion of both _software_ and _methods_.
%% - Use \proglang{}, \pkg{}, and \code{} markup throughout the manuscript.
%% - If such markup is in (sub)section titles, a plain text version has to be
%%   added as well.
%% - All software mentioned should be properly \cite-d.
%% - All abbreviations should be introduced.
%% - Unless the expansions of abbreviations are proper names (like "Journal
%%   of Statistical Software" above) they should be in sentence case

\section[Introduction]{Introduction} \label{sec:intro}

Antimicrobial resistance is a global health problem and of great concern for
human medicine, veterinary medicine, and the environment alike.  It is
associated with significant burdens to both patients and health care
systems.  Current estimates show the immense dimensions we are already
facing, such as claiming at least 50,000 lives due to antimicrobial
resistance each year across Europe and the United States alone \citep{ONeill2014-fa}. 
Although estimates for the burden through antimicrobial resistance and their
predictions are disputed \citep{De_Kraker2016-aq} the rising trend is
undeniable \citep{CDC2019}, thus calling for worldwide efforts on tackling
this problem.

Surveillance programs and reliable data are key for controlling and
streamlining these efforts.  Surveillance data of antimicrobial resistance
at higher levels (national or international) usually comprise aggregated
numbers.  The basis of this information is generated and stored at local
microbiological laboratories where isolated microorganisms are tested for
their susceptibility to a whole range of antimicrobial agents.  The efficacy
of these agents against microorganisms is nowadays interpreted as follows
\citep{eucast_newRSI}:
%
\begin{itemize}
  \item{R (``resistant'') -- there is a high likelihood of therapeutic \emph{failure};}
  %
  \item{S  (``susceptible, standard dosing regimen'') -- there is a high
likelihood of therapeutic \emph{success} using a standard dosing regimen of
an antimicrobial agent;}
  %
  \item{I (``susceptible, increased exposure'') -- there is a high likelihood
of therapeutic \emph{success}, but only when exposure to an antimicrobial
agent is increased by adjusting the dosing regimen or its concentration at
the site of infection.}
  %
\end{itemize}
%
Generally, antimicrobial resistance is defined as the proportion of
resistant microorganisms (R) among all tested microorganisms of the same
species (R + S + I).  Today, the two major guideline institutes to define
the international standards on antimicrobial resistance are the European
Committee on Antimicrobial Susceptibility Testing
\citep[EUCAST,][]{Leclercq2013-ml} and the Clinical and Laboratory Standards Institute
\citep[CLSI,][]{Clinical_and_Laboratory_Standards_Institute2014-fb}.  The
guidelines from these two institutes are adopted by 94\% of all countries
reporting antimicrobial resistance to the WHO
\citep{World_Health_Organization2018-bn}.

Although these standardized guidelines are in place on the laboratory level
for the data generation process, stored data in laboratory information
systems are often not yet suitable for data analysis.  Laboratory
information systems are often designed to fit billing purposes rather than
epidemiological data analysis.  Furthermore, (inter-)national surveillance
is hindered by inadequate standardization of epidemiological definitions,
different types of samples and data collection, settings included,
microbiological testing methods (including susceptibility testing), and data
sharing policies \citep{Tacconelli2018-tb}.  The necessity of accurate data
analysis in the field of antimicrobial resistance has just recently been
further underlined \citep{Limmathurotsakul2019}.  Antimicrobial resistance
analyses require a thorough understanding of microbiological tests and their
results, the biological taxonomy of microorganisms, the clinical and
epidemiological relevance of the results, their pharmaceutical implications,
and (inter-)national standards and guidelines for working with and reporting
antimicrobial resistance.

Here, we describe the \pkg{AMR} package \citep{Berends-AMRpackage} for \proglang{R} \citep{base_r},
which has been developed to standardize clean and reproducible antimicrobial
resistance data analyses using international standardized recommendations
\citep{Leclercq2013-ml, Clinical_and_Laboratory_Standards_Institute2014-fb}
while incorporating scientifically reliable reference data about valid
laboratory outcome, antimicrobial agents, and the complete biological
taxonomy of microorganisms.  The \pkg{AMR} package provides solutions and
support for these aspects while being independent of underlying laboratory
information systems, thereby democratizing the analysis process.  Developed
in \proglang{R} and available on the Comprehensive \proglang{R} Archive Network (CRAN)
at \url{https://CRAN.R-project.org/package=AMR}
since February 22\textsuperscript{nd}~2018 \citep{Berends-AMRpackage}, the
\pkg{AMR} package enables reproducible workflows as described in other
fields, such as environmental science \citep{Lowndes2017-hh}.  The \pkg{AMR}
package provides a new technical instrument to aid in curbing the global
threat of antimicrobial resistance.  Furthermore, local and regional data in
the laboratories can now become relevant in any setting for public health.

While no other packages \proglang{R} package with the purpose of dealing
with antimicrobial resistance data are available on CRAN or \pkg{Bioconductor},
the \pkg{AMR} package may be integrated in workflows of related packages. 
For example, the \proglang{R} Epidemics Consortium (RECON) provides high-quality
packages for data analysis in infectious disease outbreaks or epidemics 
\citep[for example \pkg{incidence} and \pkg{epicontacts},][]{incidence,epicontacts}.
In addition, on the laboratory side the \pkg{antibioticR}
package provides approaches to work with disk diffusion zone diameter and
minimum inhibitory concentration data from environment samples
\citep{antibioticR}.  We aim at providing a comprehensive and standardized
toolbox for antimicrobial resistance data processing and analysis, with a
focus on microbiological, clinical, and epidemiological purposes that was
yet missing.

The following sections describe the functionality of the \pkg{AMR} package
according to its core functionalities for transforming, enhancing, and
analyzing antimicrobial resistance data using scientifically reliable
reference data.

%% -- Manuscript ---------------------------------------------------------------

%% - In principle "as usual" again.
%% - When using equations (e.g., {equation}, {eqnarray}, {align}, etc.
%%   avoid empty lines before and after the equation (which would signal a new
%%   paragraph.
%% - When describing longer chunks of code that are _not_ meant for execution
%%   (e.g., a function synopsis or list of arguments), the environment {Code}
%%   is recommended. Alternatively, a plain {verbatim} can also be used.
%%   (For executed code see the next section.)

\section[Antimicrobial resistance data]{Antimicrobial resistance data}
\label{sec:amrdata}

Microbiological tests can be performed on different specimens, such as blood
or urine samples or nasal swabs.  After arrival at the microbiological
laboratory, the specimens are traditionally cultured on specific media, such
as blood agar.  If a microorganism can be isolated from these media, it is
tested against several antimicrobial agents.  Based on the minimal
inhibitory concentration (MIC) of the respective agent and interpretation
guidelines, such as guidelines by EUCAST \citep{Leclercq2013-ml} and CLSI
\citep{Clinical_and_Laboratory_Standards_Institute2014-fb}, test results are
reported as ``resistant''~(R), ``susceptible''~(S) or ``susceptible,
increased exposure''~(I).  A typical data structure is illustrated in
Table~\ref{tab:examplereport1} \citep{Leclercq2013-ml}.
%
\begin{table}[t!]
\centering
\begin{tabular}{cccccccc}
\hline
\code{patient} & \code{date}  & \code{test\_no} & \code{specimen} & \code{mo} & \code{PEN} & \code{AMC} & \code{CIP} \\
\hline
1      & 2019-03-08 & 100          & blood    & esccol & R   & I   & S  \\
1      & 2019-03-09 & 101          & blood    & esccol & R   & I   & S  \\
2      & 2019-03-08 & 102          & blood    & staaur & R   & S   & -  \\
3      & 2019-03-08 & 103          & urine    & pseaer & R   & R   & R  \\
\hline 
\multicolumn{8}{l}{\textnormal{\footnotesize{Abbreviations: R = resistant, S = susceptible, I = susceptible, increased exposure,}}} \\
\multicolumn{8}{l}{\textnormal{\footnotesize{mo = microorganism, PEN = penicillin, AMC = amoxicillin/clavulanic acid, CIP = ciprofloxacin}}}
\end{tabular}
\caption{Example of an antimicrobial resistance report.}
\label{tab:examplereport1}
\end{table}
%

For the first two rows, the information should be read as: \emph{Escherichia
coli} (\code{mo = esccol}) was isolated from blood of patient~1
and was found to be resistant to penicillin, and susceptible to
amoxicillin/clavulanic acid and ciprofloxacin.  However, often (especially
when merging sources) data is reported in ambiguous formats as exemplified
in Table~\ref{tab:examplereport2}.  It is crucial that source data can be
analyzed in a reliable way, especially when the outcome will be used to
evaluate patient treatment options.  This requires reproducible and
field-specific, specialized data cleaning and transforming.
%
\begin{table}[t!]
\centering
\begin{tabular}{cccccccc}
\hline
\code{patient} & \code{date}  & \code{test\_no} & \code{specimen} & \code{mo}   & \code{PEN}  & \code{AMC}  & \code{CIP}\\
\hline
1     & 2019-03-08  & 100   & blood    & esccol   & R          & I              & S\\
1     & 2019-03-09  & 101   & blood    & esccol   & R          & I              & S\\
2     & 2019-03-08  & 102   & blood    & StaAur   & $>8$ (R)*  & $<0.01$ (S)*   & .\\
3     & 2019-03-08  & 103   & urine    & P. aeru. & R          & S**            & S\\
\hline
\multicolumn{8}{l}{\footnotesize{* Mixed reporting of minimal inhibitory concentration (MIC) and susceptibility interpretation of}} \\
\multicolumn{8}{l}{\footnotesize{MIC value}}\\
\multicolumn{8}{l}{\footnotesize{** False reporting; \emph{Pseudomonas aeruginosa} (mo = P. aeru.) is intrinsically resistant to}} \\
\multicolumn{8}{l}{\footnotesize{amoxicillin/clavulanic acid (AMC)}}\\
\multicolumn{8}{l}{\footnotesize{Abbreviations: R = resistant, S = susceptible, I = susceptible, increased exposure,}} \\
\multicolumn{8}{l}{\footnotesize{mo = microorganism, PEN = penicillin, AMC = amoxicillin/clavulanic acid, CIP = ciprofloxacin}}
\end{tabular}
\caption{Antimicrobial resistance report example -- ambiguous formats.}
\label{tab:examplereport2}
\end{table}
%

The \pkg{AMR} package aims at providing a standardized and automated way of
cleaning, transforming, and enhancing these typical data structures
(Table~\ref{tab:examplereport1} and \ref{tab:examplereport2}), independent
of the underlying data source.  Processed data would be similar to
Table~\ref{tab:examplereport3} that highlights several package
functionalities in the sections below.
%
\begin{table}[t!]
\centering
% \resizebox{\textwidth}{!}{
\small
\begin{tabular}{@{}lccccccccll@{}}
\hline
\code{patient} & \code{date} & \code{test\_no} & \code{specimen} & \code{mo\textsuperscript{a}} &
\code{PEN\textsuperscript{b}} & \code{AMC\textsuperscript{b}} & \code{CIP\textsuperscript{b}} &
\code{first\_isolate\textsuperscript{c}} & \code{name\textsuperscript{d}} &
\code{gram\_stain\textsuperscript{e}} \\
\hline
1 & 2019-03-08 & 100 & blood & B\_ESCHR\_COLI & R & I & S & TRUE & Escherichia coli & Gram-negative\\
1 & 2019-03-09 & 101 & blood & B\_ESCHR\_COLI & R & I & S & FALSE & Escherichia coli & Gram-negative\\
2 & 2019-03-08 & 102 & blood & B\_STPHY\_AURS & R & S & NA & TRUE & Staphylococcus aureus & Gram-positive\\
3 & 2019-03-08 & 103 & urine & B\_PSDMN\_AERG & R & R & S & TRUE & Pseudomonas aeruginosa & Gram-negative\\
\hline
\multicolumn{11}{l}{a) \fct{as.mo} function} \\
\multicolumn{11}{l}{b) \fct{eucast\_rules} function applied} \\
\multicolumn{11}{l}{c) \fct{first\_isolate} function} \\
\multicolumn{11}{l}{d) \fct{mo\_name} function} \\
\multicolumn{11}{l}{e) \fct{mo\_gramstain} function} \\
\multicolumn{11}{l}{Abbreviations: R = resistant, S = susceptible, I = susceptible, increased exposure,} \\
\multicolumn{11}{l}{mo = microorganism, PEN = penicillin, AMC = amoxicillin/clavulanic acid, CIP = ciprofloxacin} \\
\end{tabular}
% }
\caption{Enhanced antimicrobial resistance report example.}
\label{tab:examplereport3}
\end{table}
%

\section{Antimicrobial resistance data transformation}
\label{sec:amrdatatransformation}

\subsection{Working with taxonomically valid microorganism names}

Coercing is a computational process of forcing output based on an input. 
For microorganism names, coercing user input to taxonomically valid
microorganism names is crucial to ensure correct interpretation and to
enable grouping based on taxonomic properties.  To this end, the \pkg{AMR}
package includes all microbial entries from The Catalogue of Life
(\url{http://www.catalogueoflife.org}), the most comprehensive and authoritative
global index of species currently available \citep{CoL-zq}.  It holds
essential information on the names, relationships, and distributions of more
than 1.9~million species.  The integration of it into the \pkg{AMR} package
is described in the Appendix~\ref{app:datasets}.

The \fct{as.mo} function makes use of this underlying data to transform a
vector of characters to a new class \class{mo} of taxonomically valid
microorganism name.  The resulting values are microbial IDs, which are
human-readable for the trained eye and contain information about the
taxonomic kingdom, genus, species, and subspecies (Figure~\ref{fig:mocode}).
%
\begin{figure}[t!]
\centering
\includegraphics[width=.7\textwidth]{jss3986mo_code.pdf}
\caption{\label{fig:mocode} The structure of a typical microbial ID as used
in the \pkg{AMR} package.  An ID consists of two to four elements, separated
by an underscore.  The first element is the abbreviation of the taxonomic
kingdom.  The remaining elements consist of abbreviations of the lowest
taxonomic levels of every microorganism: genus, species (if available) and
subspecies (if available).  Abbreviations used for the microbial IDs of
microorganism names were created using the base \proglang{R} function
\fct{abbreviate}.}
\end{figure}
%

The \fct{as.mo} function compares the user input with taxonomically valid
microorganism names, rates the matching with a score and returns results
based on the highest score.  This matching score ($m$), ranging from $0$ to
$1$, is calculated using the following equation:
%
\begin{equation*}
\label{eq:mo_uncertainty}
m_{(x,n)} = \frac{l_{n} - 0.5 \cdot \min\{ l_n, \operatorname{lev}(x,n) \} }{l_{n} \cdot p_{n} \cdot k_{n}}
\end{equation*}
%
where:
%
\begin{itemize}
  \item{$x$ is the user input;}
  \item{$n$ is a taxonomic name (genus, species, and subspecies);}
  \item{$l_n$ is the length of $n$;}
  \item{lev is the Levenshtein distance function \citep{Levenshtein1966},
  which counts any insertion, deletion and substitution as $1$ that is needed
  to change $x$ into $n$;}
  \item{$p_n$ is the human pathogenic prevalence group of $n$, as described
  below;}
  \item{$k_n$ is the taxonomic kingdom of $n$, set as Bacteria = 1, Fungi = 2,
  Protozoa = 3, Archaea = 4, others = 5.}
\end{itemize}
%
The grouping into human pathogenic prevalence ($p$) is based on experience
from several microbiological laboratories in the Netherlands in conjunction
with international reports on pathogen prevalence \citep{De_Greeff2019-xl,
EARS_Net, World_Health_Organization2018-bn}.  \underline{Group 1} (most
prevalent microorganisms) consists of all microorganisms where the taxonomic
class is Gammaproteobacteria or where the taxonomic genus is
\emph{Enterococcus}, \emph{Staphylococcus} or \emph{Streptococcus}.  This
group consequently contains all common Gram-negative bacteria, such as
\emph{Pseudomonas} and \emph{Legionella} and all species within the order
Enterobacterales.  \underline{Group 2} consists of all microorganisms where
the taxonomic phylum is Proteobacteria, Firmicutes, Actinobacteria or
Sarcomastigophora, or where the taxonomic genus is \emph{Absidia},
\emph{Acremonium}, \emph{Actinotignum}, \emph{Alternaria},
\emph{Anaerosalibacter}, \emph{Apophysomyces}, \emph{Arachnia},
\emph{Aspergillus}, \emph{Aureobacterium}, \emph{Aureobasidium},
\emph{Bacteroides}, \emph{Basidiobolus}, \emph{Beauveria},
\emph{Blastocystis}, \emph{Branhamella}, \emph{Borrelia},
\emph{Calymmatobacterium}, \emph{Candida}, \emph{Capnocytophaga},
\emph{Catabacter}, \emph{Chaetomium}, \emph{Chlamydia}, \emph{Chlamydophila},
\emph{Chryseobacterium}, \emph{Chryseomonas}, \emph{Chrysonilia},
\emph{Cladophialophora}, \emph{Cladosporium}, \emph{Conidiobolus},
\emph{Cryptococcus}, \emph{Curvularia}, \emph{Exophiala},
\emph{Exserohilum}, \emph{Fla\-vobacterium}, \emph{Fonsecaea},
\emph{Fusarium}, \emph{Fusobacterium}, \emph{Hendersonula},
\emph{Hypomyces}, \emph{Koserella}, \emph{Lelliottia}, \emph{Leptosphaeria},
\emph{Leptotrichia}, \emph{Malassezia}, \emph{Malbranchea},
\emph{Mortierella}, \emph{Mucor}, \emph{Mycocentrospora}, \emph{Mycoplasma},
\emph{Nectria}, \emph{Ochroconis}, \emph{Oidiodendron}, \emph{Phoma},
\emph{Piedraia}, \emph{Pithomyces}, \emph{Pityrosporum},
\emph{Prevotella},\emph{Pseudallescheria}, \emph{Rhizomucor},
\emph{Rhizopus}, \emph{Rhodotorula}, \emph{Scolecobasidium},
\emph{Scopulariopsis}, \emph{Scytalidium},\emph{Sporobolomyces},
\emph{Stachybotrys}, \emph{Stomatococcus}, \emph{Treponema},
\emph{Trichoderma}, \emph{Trichophyton}, \emph{Trichosporon},
\emph{Tritirachium} or \emph{Ureaplasma}.  \underline{Group 3} consists of
all other microorganisms.

This will lead to the effect that e.g.,~\code{"E.  coli"} will return the
microbial ID of \emph{Escherichia coli} ($m = 0.688$, a highly prevalent
microorganism found in humans) and not \emph{Entamoeba coli} ($m = 0.079$, a
less prevalent microorganism in humans), although the latter would
alphabetically come first.  The matching score function is for users
available as \fct{mo\_matching\_score}.

If any coercion rules are applied, a warning is printed to the console and
scores can be reviewed by calling \fct{mo\_uncertainties}, that prints all
other matches with their matching scores.  Users can furthermore control the
coercion rules by setting the \code{allow\_uncertain} argument in the
\fct{as.mo} function.  The following values or levels can be used:
%
\begin{itemize}
  \item{\code{0}: no additional rules are applied;}
  \item{\code{1}: allow previously accepted (but now invalid) taxonomic
  names and minor spelling errors;}
  \item{\code{2}: allow all of \code{1}, strip values between brackets, inverse
  the words of the input, strip off text elements from the end keeping at least
  two elements;}
  \item{\code{3}: allow all of level \code{1} and \code{2}, strip off text
  elements from the end, allow any part of a taxonomic name;}
  \item{\code{TRUE} (default): equivalent to \code{2};}
  \item{\code{FALSE}: equivalent to \code{0}.}
\end{itemize}
%
To support organization specific microbial IDs, users can specify a custom
reference \class{data.frame}, by using \code{as.mo(..., reference\_df =
...)}.  This process can also be automated by users with the
\fct{set\_mo\_source} function.

\subsubsection{Properties of microorganisms}

The package contains functions to return a specific (taxonomic) property of
a microorganism from the \code{microorganisms} data set (see Appendix~\ref{app:datasets}).
Functions that start with \code{mo_*} can be used to
retrieve  the most recently defined taxonomic properties of any
microorganism quickly and conveniently.  These functions rely on the
\fct{as.mo} function internally: \fct{mo\_name}, \fct{mo\_fullname},
\fct{mo\_shortname}, \fct{mo\_subspecies}, \fct{mo\_species},
\fct{mo\_genus}, \fct{mo\_family}, \fct{mo\_order}, \fct{mo\_class},
\fct{mo\_phylum}, \fct{mo\_kingdom}, \newline\fct{mo\_type}, \fct{mo\_gramstain},
\fct{mo\_ref},\fct{mo\_authors}, \fct{mo\_year},
\fct{mo\_rank},\newline\fct{mo\_taxonomy}, \fct{mo\_synonyms}, \fct{mo\_info} and
\fct{mo\_url}.  Determination of the Gram stain, by using
\fct{mo\_gramstain}, is based on the taxonomic subkingdom and phylum. 
According to \cite{Cavalier-Smith2002}, who defined the subkingdoms
Negibacteria and Posibacteria, only the following phyla are Posibacteria:
Actinobacteria, Chloroflexi, Firmicutes and Tenericutes.  Bacteria from
these phyla are considered Gram-positive -- all other bacteria are considered
Gram-negative.  Gram stains are only relevant for species within the kingdom
of Bacteria.  For species outside this kingdom, \fct{mo\_gramstain} will
return \code{NA}.

\subsection{Working with antimicrobial names or codes}

The \pkg{AMR} package includes the \code{antibiotics} data set, which
comprises common laboratory information system codes, official names, anatomical therapeutic chemical
(ATC) codes, defined daily doses (DDD) and more
than 5,000 trade names of 464 antimicrobial agents (see Appendix~\ref{app:datasets}).
The ATC code system and the reference list for DDDs
have been developed and made available by the World Health Organization
Collaborating Centre for Drug Statistics Methodology (WHOCC) to standardize
pharmaceutical classifications
\citep{WHO_Collaborating_Centre_for_Drug_Statistics_Methodology2018-bj}. 
All agents in the \code{antibiotics} data set have a unique antimicrobial
ID, which is based on abbreviations used by the European Antimicrobial
Resistance Surveillance Network (EARS-Net), the largest publicly funded
system for antimicrobial resistance surveillance in Europe
\citep{European_Centre_for_Disease_Prevention_and_Control2018-xc}. 
Furthermore, the \pkg{AMR} package includes the \code{antivirals} data set
containing antiviral agents, which is also described in the Appendix~\ref{app:datasets}.

\subsubsection{Properties of antimicrobial agents}

It is a common task in microbiological data analyses (and other clinical or
epidemiological fields) to work with different antimicrobial agents.  The
\pkg{AMR} package provides several functions to translate inputs such as ATC
codes, abbreviations, or names in any direction.  Using \fct{as.ab}, any
input will be transformed to an antimicrobial ID of class \class{ab}. 
Helper functions are available to get specific properties of antimicrobial
IDs, such as \fct{ab\_name} for getting the official name, \fct{ab\_atc} for
the ATC code, or \fct{ab\_cid} for the compound ID (CID) used by PubChem
\citep{Kim2019-so}.  Trade names can be also used as input.  For example,
the input values \code{"Amoxil"}, \code{"dispermox"}, \code{"amox"} and
\code{"J01CA04"} all return the ID of amoxicillin (\code{AMX}):
%
\begin{CodeChunk}
\begin{CodeInput}
R> as.ab("Amoxicillin")
\end{CodeInput}
\begin{CodeOutput}
Class <ab>
[1] AMX
\end{CodeOutput}
\begin{CodeInput}
R> as.ab(c("Amoxil", "dispermox", "amox", "J01CA04"))
\end{CodeInput}
\begin{CodeOutput}
Class <ab>
[1] AMX AMX AMX AMX
\end{CodeOutput}
\begin{CodeInput}
R> ab_name("Amoxil")
\end{CodeInput}
\begin{CodeOutput}
[1] "Amoxicillin"
\end{CodeOutput}
\begin{CodeInput}
R> ab_atc("amox")
\end{CodeInput}
\begin{CodeOutput}
[1] "J01CA04"
\end{CodeOutput}
\begin{CodeInput}
R> ab_name("J01CA04")
\end{CodeInput}
\begin{CodeOutput}
[1] "Amoxicillin"
\end{CodeOutput}
\end{CodeChunk}
%
If more than one antimicrobial agent is found in the input string, a warning
with the additional findings is printed to the console.

\subsubsection{Selecting and filtering data based on classes of antimicrobial agents}

The application of the ATC classification system also enables grouping of
antimicrobial agents for data analyses.  Data sets with microbial isolates
can be filtered on isolates with specific results for tested antimicrobial
agents in a specific antimicrobial class.  For example, \code{carbapenems()}
can be used to select columns or filter rows based on any of the 14 available
antimicrobial agents in the group of carbapenems according to the
\code{antibiotics} data set.

\subsection{Working with antimicrobial susceptibility test results}

Minimal inhibitory concentrations (MIC) are susceptibility test results
measured by microbiological laboratory equipment to determine at which
minimum antimicrobial drug concentration 99.9\% of a microorganism is
inhibited in growth.  These concentrations are often capped at a minimum and
maximum, for example $\leq$\code{0.02 µg/ml} and $\geq$\code{32 µg/ml},
respectively.  The \class{mic} class, an ordered \class{factor} containing
valid MIC values, keeps these operators while still ordering all possible
outcomes correctly so that e.g.,~\code{"<= 0.02"} will be considered lower
than \code{"0.04"}.

Another susceptibility testing method is the use of drug diffusion disks,
which are small tablets containing a specified concentration of an
antimicrobial agent.  These disks are applied onto a solid growth medium or
a specific agar plate.  After 24 hours of incubation, the diameter of the
growth inhibition around a disk can be measured in millimeters with a ruler. 
The \class{disk} class can be used to clean these kinds of measurements,
since they should always be valid numeric values between 6 and 50.  The
supported minima and maxima of valid values for both classes, \class{mic}
and \class{disk}, are displayed in Table~\ref{tab:susceptclasses}.
%
\begin{table}[t!]
\centering
\begin{tabular}{llll}
\hline
Class & Minimum & Maximum & Unit \\ 
\hline
\class{mic}  & $\leq 0.001$ & $\geq 1024$ & µg/ml \\
\class{disk} & $\leq 6$ & $\geq 50$ & mm \\ 
\hline
\end{tabular}
\caption{Antimicrobial suceptibility test classes.}
\label{tab:susceptclasses}
\end{table}
%

The higher the MIC or the smaller the growth inhibition diameter, the more
active substance of an antimicrobial agent is needed to inhibit cell growth,
i.e.,~the higher the antimicrobial resistance against the tested
antimicrobial agent.  At high MICs and small diameters, guidelines interpret
the microorganism as ``resistant''~(R) to the tested antimicrobial agent. 
At low MICs and wide diameters, guidelines interpret the microorganism as
``susceptible''~(S) to the tested antimicrobial agent.  In between, the
microorganism is classified as ``susceptible, increased exposure''~(I).  For
these three interpretations the \class{rsi} class has been developed.  When
using \fct{as.rsi} on MIC values (of class \class{mic}) or disk diffusion
diameters (of class \class{disk}), the values will be interpreted according
to the guidelines from the CLSI or EUCAST 
\citep[][all guidelines between 2011 and
2020 are included in the \pkg{AMR} package]{Clinical_and_Laboratory_Standards_Institute2019-tu,
The_European_Committee_on_Antimicrobial_Susceptibility_Testing_undated-hs}. 
Guidelines can be changed by setting the \code{guidelines} argument.
%
\begin{CodeChunk}
\begin{CodeInput}
R> # Low MIC value
R> as.rsi(as.mic(2), "E. coli", "ampicillin", guideline = "EUCAST 2020")
\end{CodeInput}
\begin{CodeOutput}
Class <rsi>
[1] S
\end{CodeOutput}
\begin{CodeInput}
R> # High MIC value
R> as.rsi(as.mic(32), "E. coli", "ampicillin", guideline = "EUCAST 2020")
\end{CodeInput}
\begin{CodeOutput}
Class <rsi>
[1] R
\end{CodeOutput}
\end{CodeChunk}
%
When using the \fct{as.rsi} function on existing antimicrobial
interpretations, it tries to coerce the input to the values ``R'', ``S'' or
``I''.  These values can in turn be used to calculate the proportion of
antimicrobial resistance.

\subsection{Interpretative rules by EUCAST}

Next to supplying guidelines to interpret raw MIC values, the EUCAST has
developed a set of expert rules to assist clinical microbiologists in the
interpretation and reporting of antimicrobial susceptibility tests
\citep{Leclercq2013-ml}.  The rules comprise assistance on intrinsic
resistance, exceptional phenotypes, and interpretive rules.  The \pkg{AMR}
package covers intrinsic resistant and interpretive rules for data
transformation and standardization purposes.  The first prevents false
susceptibility reporting by providing a list of organisms with known
intrinsic resistance to specific antimicrobial agents (e.g.,~cephalosporin
resistance of all enterococci).  Interpretative rules apply inference from
established resistance mechanisms \citep{Winstanley2011-vh,
Courvalin1992-ng, Courvalin1996-fw, Livermore2001-tf}.  Both groups of rules
are based on classic IF THEN statements (e.g.,~IF \emph{Enterococcus spp.}
resistant to ampicillin THEN also report as resistant to imipenem).  Some
rules provide assistance for further actions when certain resistance has
been detected, i.e.,~performing additional testing of the isolated
microorganism.  The \pkg{AMR} package function \fct{eucast\_rules} can apply
all EUCAST rules that do not rely on additional clinical information, such
as additional information on patients' diagnoses. 
Table~\ref{tab:examplereport2} and \ref{tab:examplereport3} highlight the
transformation for the reporting of AMX = S in patient\_id = 000003 to the
correct report according to EUCAST rules of AMX = R.  Of note, however,
EUCAST rules overwrite original data to correct for the difference in how
antimicrobial agents affect the tested microorganism \emph{in vitro} (in the
laboratory) and \emph{in vivo} (in the human body).  This requires users to
closely collaborate with the data source provider to ensure correct
versioning, backward compatibility, reproducibility, and taking into account
specific local regulation for resistance reporting.  Typical scenarios where
changes to the original data points apply include \emph{in vitro} test
results indicating susceptibility when resistance \emph{in vivo} is known. 
The changes are based on scientific consensus to ensure reliable
high-quality reporting of antimicrobial susceptibility results.  All changes
to the data are printed to the console and can also be reviewed in detail by
setting the argument \code{eucast\_rules(..., verbose = TRUE)}.

EUCAST rules are subject to regular updates which are implemented into the
\pkg{AMR} package by the \pkg{AMR} maintenance team shortly after
publication.  The \fct{eucast\_rules} function supports versioning of the
rules.  The arguments \code{version\_breakpoints} and
\newline\code{version\_expertrules} can be set to current or previous versions. 
By default, the \newline\fct{eucast\_rules} function uses the latest implemented
version.

\subsection{Working with defined daily doses (DDD)}

DDDs are essential for standardizing antimicrobial consumption analysis, for
inter-institutional or international comparison.  The official DDDs are
published by the WHOCC
\citep{WHO_Collaborating_Center_for_Drug_Statistics_Methodology_undated-sl}. 
Updates to the official publication are monitored by the \pkg{AMR}
maintenance team and implemented in the \code{antibiotics} data set included
in the \pkg{AMR} package.  Other metrics exist such as the recommended daily
dose (RDD) or the prescribed daily dose (PDD).  However, DDDs are the only
metric that is independent of a patient's disease and therapeutic choices
and thus suitable for the \pkg{AMR} package.

Functions from the \fct{atc\_online\_*} family take any text as input that
can be coerced with \fct{as.ab} (i.e.,~to class \class{ab}).  Next, the
functions access the WHOCC online registry
\citep[][internet connection required]{WHO_Collaborating_Center_for_Drug_Statistics_Methodology_undated-sl}
and download the property defined in the
arguments (e.g.,~\code{administration = "O"} or \code{administration = "P"}
for oral or parenteral administration and \code{property = "ddd"} or
\code{property = "groups"} to get DDD or the group of the selected
antimicrobial defined by its ATC code).
%
\begin{CodeChunk}
\begin{CodeInput}
R> atc_online_ddd("amoxicillin", administration = "O")
\end{CodeInput}
\begin{CodeOutput}
[1] 1.5
\end{CodeOutput}
\begin{CodeInput}
R> atc_online_groups("amoxicillin")
\end{CodeInput}
\begin{CodeOutput}
[1] "ANTIINFECTIVES FOR SYSTEMIC USE"        
[2] "ANTIBACTERIALS FOR SYSTEMIC USE"        
[3] "BETA-LACTAM ANTIBACTERIALS, PENICILLINS"
[4] "Penicillins with extended spectrum"     
\end{CodeOutput}
\end{CodeChunk}
%

\section{Enhancing antimicrobial resistance data} \label{sec:enhancing}

\subsection{Determining first isolates} \label{firstisolates}

Determining antimicrobial resistance or susceptibility can be done for a
single agent (mono-therapy) or multiple agents (combination therapy).  The
calculation of antimicrobial resistance statistics is dependent on two
prerequisites: the data should only comprise the first isolates and a
minimum required number of 30 isolates should be met for every stratum in
further analysis \citep{Clinical_and_Laboratory_Standards_Institute2014-fb}.

An isolate is a microorganism strain cultivated on specified growth media in
a laboratory, so its phenotype can be determined.  First isolates are
isolates of any species found first in a patient per episode, regardless of
the body site or the type of specimen (such as blood or urine)
\citep{Clinical_and_Laboratory_Standards_Institute2014-fb}.  The selection
on first isolates (using function \fct{first\_isolate}) is important to
prevent selection bias, as it would lead to overestimated or underestimated
resistance to an antimicrobial agent.  For example, if a patient is admitted
with a multi-drug resistant microorganism and that microorganism is found in
five different blood cultures the following week, it would overestimate
resistance if all isolates were to be included in the analysis.  The episode
in days can be set with the argument \code{episode_days}, which defaults to
365 as suggested by the
\cite{Clinical_and_Laboratory_Standards_Institute2014-fb} guideline.

\subsection{Determining multi-drug resistant organisms (MDRO)}

Definitions of multi-drug resistant organisms (MDRO) are regulated by
national and international expert groups and differ between nations.  The
\pkg{AMR} package provides the functionality to quickly identify MDROs in a
data set using the \fct{mdro} function.  Guidelines can be set with the
argument \code{guideline}.  At default, it applies the guideline as proposed
by \cite{Magiorakos2012}.  Their work describes the definitions for bacteria
being ``MDR''~(multi-drug-resistant), ``XDR''~(extensively drug-resistant) or
``PDR''~(pan-drug-resistant).  These definitions are widely adopted
\citep{Abat2018} and known in the field of medical microbiology.

Other guidelines currently supported are the international EUCAST guideline
(\code{guideline = "EUCAST"},
\cite{European_Committee_on_Antimicrobial_Susceptibility_Testing_EUCAST_undated-ny}),
the international WHO guideline on the management of drug-resistant
tuberculosis,
\citep[\code{guideline = "TB"},][]{World_Health_Organization_TB-ji}, and the national guidelines of The
Netherlands
\citep[\code{guideline = "NL"},][]{Werkgroep_Infectiepreventie_WIP_undated-vu}, and Germany
\citep[\code{guideline = "DE"},][]{Mueller2015}.

Some guidelines require a minimum availability of tested antimicrobial
agents per isolate.  This is needed to prevent false-negatives, since no
reliable determination can be performed on only a few test results.  This
required minimum defaults to 50\%, but can be set by the user with the
\code{pct_minimum_classes}.  Isolates that do not meet this requirement will
be skipped for determination and will return \code{NA} (not applicable),
with an informative warning printed to the console.

The rules are applied per row of the data.  The \fct{mdro} function
automatically identifies the variables containing the microorganism codes
and antimicrobial agents based on the \fct{guess\_ab\_col} function. 
Following the guideline set by the user, it analyzes the specific
antimicrobial resistance of a microorganism and flags that microorganism
accordingly.  The outcome is demonstrated in Table~\ref{tab:mdro}, where the
first row is an MDRO according to the Dutch guidelines
\citep{Werkgroep_Infectiepreventie_WIP_undated-vu}.
%
\begin{table}[t!]
\centering
\begin{tabular}{lcccccl}
\hline
\code{mo}             & \code{AMC} & \code{GEN} & \code{TOB} & \code{CIP} & \code{MFX} & \code{MDRO}    \\
\hline
B\_ESCHR\_COLI & S   & R   & R   & R   & R   & Positive \\
B\_ESCHR\_COLI & R   & S   & R   & R   & S   & Negative \\
B\_ESCHR\_COLI & S   & S   & S   & R   & S   & Negative \\
\hline
\multicolumn{7}{l}{\footnotesize{Abbreviations: mo = microorganism,
AMC = amoxicilline/clavulanic acid}} \\
\multicolumn{7}{l}{\footnotesize{GEN = gentamicin, TOB = tobramycin,
CIP = ciprofloxacin, MFX = moxifloxacin,}} \\
\multicolumn{7}{l}{\footnotesize{MDRO = multi-drug resistant organism,
B\_ESCHR\_COLI = microorganism code of}} \\
\multicolumn{7}{l}{\footnotesize{\emph{Escherichia coli}}}\\
\end{tabular}
\caption{Example of a multi-drug resistant organism (MDRO) in a data set
identified by applying Dutch guidelines.}
\label{tab:mdro}
\end{table}
%

The returned value is an ordered \class{factor} with the levels
\code{Negative} < \code{Positive, unconfirmed} < \code{Positive}.  For
some guideline rules that require additional testing (e.g.,~molecular
confirmation), the level \code{Positive, unconfirmed} is returned.

\subsubsection{Multi-drug resistant tuberculosis}

Tuberculosis is a major threat to global health caused by
\emph{Mycobacterium tuberculosis} (MTB) and is one of the top ten causes of
death worldwide \citep{World_Health_Organization2018-mk}.  Exceptional
antimicrobial resistance in MTB is therefore of special interest.  To this
end, the international WHO guideline for the classification of drug
resistance in MTB \citep{World_Health_Organization_TB-ji} is included in the
\pkg{AMR} package.  The \fct{mdr\_tb} function is a convenient wrapper
around \code{mdro(..., guideline = "TB")}, which returns an other ordered
\class{factor} than other \fct{mdro} functions.  The output will contain the
\class{factor} levels \code{Negative} < \code{Mono-resistant} <
\code{Poly-resistant} < \code{Multi-drug-resistant} < \code{Extensive
drug\-resistant}, following the WHO guideline.

\section{Analyzing antimicrobial resistance data} \label{sec:analyzing}

\subsection{Calculation of antimicrobial resistance}
\label{sec:analyzingcalc}

The \pkg{AMR} package contains several functions for fast and simple
resistance calculations of bacterial or fungal isolates.  A minimum number
of available isolates is needed for the reliability of the outcome.  The
CLSI guideline suggests a minimum of 30 available first isolates
irrespective of the type of statistical analysis
\citep{Clinical_and_Laboratory_Standards_Institute2014-fb}.  Therefore, this
number is used as the default setting for any function in the package that
calculates antimicrobial resistance or susceptibility, which can be changed
with the \code{minimum} argument in all applicable functions.

\subsubsection{Counts}

The \pkg{AMR} package relies on the concept of tidy data
\citep{Wickham2014}, although not strictly following its rules (one row per
test rather than one row per observation).  Function names to calculate the
number of available isolates follow these general resistance interpretation
standards with \fct{count\_S}, \fct{count\_I}, and \fct{count\_R}
respectively.  Combinations of antimicrobial resistance interpretations can
be counted with \fct{count\_SI} and \fct{count\_IR}.  All these functions
take a vector of interpretations of the class \class{rsi} (as discussed
above) or are internally transformed with \fct{as.rsi}.  The returned value
is the sum of the respective interpretation in the selected test column. 
All \fct{count\_*} functions support quasi-quotation with pipes, grouped
variables, and can be used with \fct{dplyr::summarize}
\citep{Wickham_dplyr-rm}.

\subsubsection{Proportions}

Calculation of antimicrobial resistance is carried out by counting the
number of first resistant isolates (interpretation of ``R'') and dividing it
by the number of all first isolates, see Equation~\ref{eq:proportion}.  This
is implemented in the \fct{proportion\_R} function.  To calculate
antimicrobial \emph{susceptibility}, the number of susceptible first
isolates (interpretation of ``S'' and ``I'') has to be counted and divided
by the number of all first isolates, which is implemented in the
\fct{proportion\_SI} function.  For convenience, the \fct{resistance}
function is an alias of the \fct{proportion\_R} function, and the
\fct{susceptibility} function is an alias of the \fct{proportion\_SI}
function.

The functions \fct{proportion\_R}, \fct{proportion\_IR},
\fct{proportion\_I}, \fct{proportion\_SI}, and \fct{proportion\_S} follow
the same logic as the \fct{count\_*} functions and all return a vector of
class \class{double} with a value between \code{0} and \code{1}.  The
argument \code{minimum} defines the minimal allowed number of available
(tested) isolates (default: \code{minimum = 30}).  Any number below the set
\code{minimum} will return \code{NA} with a warning.

For calculating the proportion ($P$) of antimicrobial resistance or
susceptibility to one antimicrobial agent, the following equation is used:
%
\begin{equation}
\label{eq:proportion}
P_{(x, o)} = \frac{
\sum_{i=1}^k [x_i \in o]
}{
\sum_{i=1}^k [x_i \in \{R,S,I\}]
},
\end{equation}
%
where $P$ is the proportion of outcome $o$ (that is either ``R'', ``S'',
``I'', or a combination of two of them), where $x$ is a character vector of
length $k$ only consisting of values ``R'', ``S'', or ``I'' and $[x_i \in
o]$ is the indicator function, returning $1$ if the indicator function is
true and $0$ otherwise.  The denominator must include the collection $\{R,
S, I\}$ so that ’wrong’ elements in $x$ (i.e.,~elements not being ``R'',
``S'', or ``I'') will not be counted.  Thus, the theoretical antimicrobial
susceptibility of the vector $x = \{S, S, I, R, R\}$ is:
%
\begin{equation*}
\label{eq:proportion_calc}
P_{(x, o = \{S, I\})} = \frac{3}{5} = 0.6
\end{equation*}
%
For the proportion of empiric susceptibility ($s$) for more than one
antimicrobial agent, the calculation can be carried out in two ways
(Table~\ref{tab:only_all_tested}).  The first method is to count the total
number of first isolates where at least one agent was tested as ``S'' or
``I'' and divide it by the number of first isolates tested where any of the
agents was tested (Equation~\ref{eq:without_all_tested}).  This method will
be used when setting \code{only\_all\_tested = FALSE} in the
\fct{susceptibility} function:
%
\begin{equation}
\label{eq:without_all_tested}
s_{(x, y)} = \frac{
\sum_{i=1}^k [x_i \in \{S,I\} \lor y_i \in \{S,I\}]
}{
\sum_{i=1}^k [x_i \in \{R,S,I\} \lor y_i \in \{R,S,I\}]
},
\end{equation}
%
where $x$ is a character vector only consisting of values ``R'', ``S'', or
``I''~(i.e.,~``agent A'') and $y$ is another character vector only consisting
of values ``R'', ``S'', or ``I''~(i.e.,~``agent B'').

The second method is to count the total number of first isolates where at
least one agent was tested as ``S'' or ``I'' \emph{and} where all agents
were tested divided by the number of first isolates tested where \emph{all}
of the agents were tested (Equation~\ref{eq:with_all_tested}).  This method
will be used when setting \code{only\_all\_tested = TRUE} in the
\fct{susceptibility} function:
%
\begin{equation}
\label{eq:with_all_tested}
s'_{(x, y)} = \frac{
\sum_{i=1}^k [(x_i \in \{S,I\} \lor y_i \in \{S,I\}) \, \land x_i \in \{R,S,I\} \land y_i \in \{R,S,I\}]
}{
\sum_{i=1}^k [x_i \in \{R,S,I\} \land y_i \in \{R,S,I\}]
}
\end{equation}
%
\begin{table}[t!]
\centering
% \resizebox{\textwidth}{!}{
\begin{tabular}{ccccccc}
\hline
\multicolumn{2}{c}{\multirow{2}{*}{Antimicrobial agent}} &
\multicolumn{2}{c}{All isolates} &
\multicolumn{2}{c}{Only isolates tested for both agents}\\
\multicolumn{2}{c}{} & \multicolumn{2}{c}{(\code{only\_all\_tested = FALSE})} &
\multicolumn{2}{c}{(\code{only\_all\_tested = TRUE})}\\
\hline
\multirow{2}{*}{Agent A} & \multirow{2}{*}{Agent B} & Include as & Include as &
Include as & Include as\\
& & numerator & denominator & numerator & denominator\\
\hline
 S or I  &  S or I    &   \checkmark   &   \checkmark   &   \checkmark   &   \checkmark\\
   R     &   S or I   &   \checkmark   &   \checkmark   &   \checkmark   &   \checkmark\\
  N/A    &  S or I    &   \checkmark   &   \checkmark   &       &    \\
 S or I  &     R      &   \checkmark   &   \checkmark   &   \checkmark   &   \checkmark\\
   R     &     R      &       &   \checkmark   &       &   \checkmark\\
  N/A    &     R      &       &       &       &    \\
 S or I  &    N/A     &   \checkmark   &   \checkmark   &       &    \\
   R     &    N/A     &       &       &       &    \\
  N/A    &    N/A     &       &       &       &    \\
\hline
\multicolumn{6}{l}{\footnotesize{Abbreviations: R = resistant, S = susceptible,
I = susceptible/increased exposure, N/A = not tested/missing}}
\end{tabular}
% }
\caption{Example calculation for determining empiric susceptibility (\%SI) for
more than one antimicrobial agent.}
\label{tab:only_all_tested}
\end{table}
%

Based on Equation~\ref{eq:proportion}, the overall resistance and
susceptibility of antimicrobial agents like gentamicin (GEN) and amoxicillin
(AMX) can be calculated using the following syntax.  The
\code{example\_isolates} is an example data set included in the \pkg{AMR}
package, see Appendix~\ref{app:datasets}.  The \fct{n\_rsi} function is
analogous to the \fct{n} function of the \pkg{dplyr} package.  It counts the
number of available isolates, but only includes observations with valid
antimicrobial results (i.e.,~``R'', ``S'', or ``I'').
%
\begin{CodeChunk}
\begin{CodeInput}
R> library("dplyr")
R> example_isolates %>% summarize(
+    r_gen = proportion_R(GEN), r_amx = proportion_R(AMX),
+    n_gen = n_rsi(GEN), n_amx = n_rsi(AMX), n_total = n())
\end{CodeInput}
\begin{CodeOutput}
      r_gen     r_amx n_gen n_amx n_total
1 0.2458221 0.5955556  1855  1350    2000
\end{CodeOutput}
\end{CodeChunk}
%
This output reads: the resistance to gentamicin of all isolates in the
\code{example\_isolates} data set is $P{(x = GEN, o = \{R\})} = 24.6\%$,
based on $1855$ out of $2000$ available isolates.  This means that the
susceptibility is $P{(x = GEN, o = \{S,I\})} = 75.4\%$.  The susceptibility
to amoxicillin is $P{(x = AMX, o = \{S,I\})} = 40.4\%$ based on $1350$
isolates.

To calculate the effect of combination therapy, i.e.,~treating patients with
multiple agents at the same time, all \fct{proportion\_*} functions can
handle multiple variables as arguments as defined in
Equation~\ref{eq:without_all_tested} and \ref{eq:with_all_tested}.  For
example, to calculate the empiric susceptibility of a combination therapy
comprising gentamicin (GEN) and amoxicillin (AMX):
%
\begin{CodeChunk}
\begin{CodeInput}
R> example_isolates %>% summarize(
+    si_gen_amx = proportion_SI(GEN, AMX), n_gen_amx = n_rsi(GEN, AMX),
+    n_total = n())
\end{CodeInput}
\begin{CodeOutput}
  si_gen_amx n_gen_amx n_total
1   0.931843      1921    2000
\end{CodeOutput}
\end{CodeChunk}
%
This leads to the conclusion that combining gentamicin with amoxicillin
would cover \\$s{(x = GEN, y = AMX)} = 93.2\%$ based on $1921$ out of $2000$
available isolates, which is $17.8\%$ more than when treating with
gentamicin alone ($P{(x = GEN, o = \{S,I\})} = 75.4\%$).  With these
functions, exact calculations can be done to evaluate the empiric success of
treating infections with one or more antimicrobial agents.

\section{Design decisions} \label{sec:design}

The \pkg{AMR} package follows the rationale of \pkg{tidyverse} packages as
authored by \cite{Wickham2019}.  Most functions take a \class{data.frame} or
\class{tibble} as input, support piping (\code{\%>\%}) operations, can work
with quasi-quotations, and can be integrated into \pkg{dplyr} workflows,
such as \fct{mutate} to create new variables and \fct{group\_by} to group by
variables.  Although the \pkg{AMR} package integrates well into
\pkg{tidyverse} workflows, it can also be used with base \proglang{R} only. 
To this extent, the \pkg{AMR} package was developed to be independent of any
other \proglang{R} package to ensure and maintain sustainability.

The \pkg{AMR} package supports multiple languages.  Currently supported
languages are English, Dutch, French, German, Italian, Portuguese, and
Spanish.  The system language will be used if the language is supported but
can be overwritten with \code{options(AMR_locale = ...)}.  Multi-language
support affects language-dependent output of functions such as
\fct{mo\_name}, \fct{mo\_gramstain}, \fct{mo\_type}, and \fct{ab\_name}.

The \pkg{AMR} package uses \proglang{S}3 classes, object oriented systems
available in \proglang{R}.  They allow different types of output based on
the user input.  The \pkg{AMR} package introduces 5 \proglang{S}3 classes (\class{mo},
\class{ab}, \class{rsi}, \class{mic}, and \class{disk}) to increase the
convenience when working with antimicrobial susceptibility data.

\section{Reproducible example} \label{sec:reprex}

We consider the problem of working with antimicrobial resistance data from
three different hospitals between 2011-01-01 and 2020-01-01.  After loading
the \pkg{AMR} package and additional \pkg{tidyverse} packages to allow
transformation and plotting, we load the\newline\code{example_isolates_unclean}
example data from the \pkg{AMR} package into the global environment and
assign it a new name.
%
\begin{CodeChunk}
\begin{CodeInput}
R> library("dplyr")
R> library("tidyr")
R> library("AMR")
R> options(AMR_locale = "en")
R> data <- example_isolates_unclean
R> glimpse(data)
\end{CodeInput}
\begin{CodeOutput}
Rows: 3,000
Columns: 8
$ patient_id <chr> "J3", "R7", "P3", "P10", "B7", "W3", "J8", "M3", …
$ hospital   <chr> "A", "A", "A", "A", "A", "A", "A", "A", "A", "A",…
$ date       <date> 2012-11-21, 2018-04-03, 2014-09-19, 2015-12-10, …
$ bacteria   <chr> "E. coli", "K. pneumoniae", "E. coli", "E. coli",…
$ AMX        <chr> "R", "R", "R", "S", "S", "R", "R", "R", "S", "S",…
$ AMC        <chr> "I", "I", "S", "I", "S", "S", "S", "S", "S", "S",…
$ CIP        <chr> "S", "S", "S", "S", "S", "R", "S", "S", "S", "S",…
$ GEN        <chr> "S", "S", "S", "S", "S", "S", "S", "S", "S", "S",…
\end{CodeOutput}
\begin{CodeInput}
R> unique(data$hospital)
\end{CodeInput}
\begin{CodeOutput}
[1] "A" "B" "C"
\end{CodeOutput}
\begin{CodeInput}
R> unique(data$bacteria)
\end{CodeInput}
\begin{CodeOutput}
 [1] "E. coli"                  "K. pneumoniae"           
 [3] "S. aureus"                "S. pneumoniae"           
 [5] "klepne"                   "strpne"                  
 [7] "esccol"                   "staaur"                  
 [9] "Escherichia coli"         "Staphylococcus aureus"   
[11] "Streptococcus pneumoniae" "Klebsiella pneumoniae"   
\end{CodeOutput}
\begin{CodeInput}
R> data %>% count(bacteria)
\end{CodeInput}
\begin{CodeOutput}
                   bacteria   n
1                   E. coli 494
2                    esccol 508
3          Escherichia coli 516
4             K. pneumoniae 108
5     Klebsiella pneumoniae 102
6                    klepne 116
7                 S. aureus 247
8             S. pneumoniae 151
9                    staaur 240
10    Staphylococcus aureus 243
11 Streptococcus pneumoniae 139
12                   strpne 136
\end{CodeOutput}
\end{CodeChunk}
%
The data contains $3,000$ observations of 8 variables from 3 hospitals.  The
\code{bacteria} variable comprises 12 unique elements.  However, they appear to
encode the same information in different formats (\code{"E.  coli"}, \code{"K. pneumoniae"},
\code{"S.  aureus"}, \code{"S.  pneumoniae"}, \code{"klepne"}, \code{"strpne"},
\code{"esccol"}, \code{"staaur"}, \code{"Escherichia coli"},
\code{"Staphylococcus aureus"}, \newline\code{"Streptococcus pneumoniae"},
\code{"Klebsiella pneumoniae"}).  We can use the \fct{as.mo} function
to standardize the bacterial codes and add a variable with the official
scientific name.  The correct transformation of the bacterial codes can be
reviewed by calling the \fct{mo\_uncertainties} function.
%
\begin{CodeChunk}
\begin{CodeInput}
R> data <- data %>% mutate(
+    bacteria = as.mo(bacteria), bacteria_name = mo_name(bacteria))
R> mo_uncertainties()
\end{CodeInput}
\begin{CodeOutput}
Matching scores are based on human pathogenic prevalence and the
resemblance between the input and the full taxonomic name. See
`?mo_matching_score`.

"E. coli" -> Escherichia coli (B_ESCHR_COLI, 0.688)
Also matched: Entamoeba coli (0.079)
"K. pneumoniae" -> Klebsiella pneumoniae (B_KLBSL_PNMN, 0.786)
Also matched: Klebsiella pneumoniae ozaenae (0.707), Klebsiella
              pneumoniae pneumoniae (0.688) and Klebsiella pneumoniae
              rhinoscleromatis (0.658)
"S. aureus" -> Staphylococcus aureus (B_STPHY_AURS, 0.690)
Also matched: Staphylococcus aureus aureus (0.643), Streptomyces
              aureus (0.355) and Stentor aureus (0.052)

"S. pneumoniae" -> Streptococcus pneumoniae (B_STRPT_PNMN, 0.750)
Also matched: Spirabiliibacterium pneumoniae (0.700)
\end{CodeOutput}
\begin{CodeInput}
R> data %>% count(bacteria, bacteria_name)
\end{CodeInput}
\begin{CodeOutput}
      bacteria            bacteria_name    n
1 B_ESCHR_COLI         Escherichia coli 1518
2 B_KLBSL_PNMN    Klebsiella pneumoniae  326
3 B_STPHY_AURS    Staphylococcus aureus  730
4 B_STRPT_PNMN Streptococcus pneumoniae  426
\end{CodeOutput}
\end{CodeChunk}
%
In a next step, we can further enrich the data with additional microbial
taxonomic data based on the \code{bacteria} variable, such as Gram-stain and
microorganism family.
%
\begin{CodeChunk}
\begin{CodeInput}
R> data <- data %>% mutate(
+    gram_stain = mo_gramstain(bacteria), family = mo_family(bacteria))
R> data %>% count(gram_stain)
\end{CodeInput}
\begin{CodeOutput}
     gram_stain    n
1 Gram-negative 1844
2 Gram-positive 1156
\end{CodeOutput}
\begin{CodeInput}
R> data %>% count(family)
\end{CodeInput}
\begin{CodeOutput}
              family    n
1 Enterobacteriaceae 1844
2  Staphylococcaceae  730
3   Streptococcaceae  426
\end{CodeOutput}
\end{CodeChunk}
%
The variables \code{AMX}, \code{AMC}, \code{CIP}, and \code{GEN} contain antimicrobial
susceptibility test results.  The abbreviations stand for the tested
antimicrobial agent.  The official names and additional information about
the antimicrobial agents can be checked with the \fct{ab\_info} function
from the \pkg{AMR} package.
%
\begin{CodeChunk}
\begin{CodeInput}
R> ab_info("AMX")
\end{CodeInput}
\begin{CodeOutput}
$ab
[1] "AMX"

$cid
[1] 33613

$name
[1] "Amoxicillin"

$group
[1] "Beta-lactams/penicillins"

$atc
[1] "J01CA04"

$atc_group1
[1] "Beta-lactam antibacterials, penicillins"

$atc_group2
[1] "Penicillins with extended spectrum"

$tradenames
 [1] "actimoxi"           "amoclen"            "amolin"            
 [4] "amopen"             "amopenixin"         "amoxibiotic"       
 [7] "amoxicaps"          "amoxicilina"        "amoxicillin"       
[10] "amoxicilline"       "amoxicillinum"      "amoxiden"          
[13] "amoxil"             "amoxivet"           "amoxy"             
[16] "amoxycillin"        "anemolin"           "aspenil"           
[19] "biomox"             "bristamox"          "cemoxin"           
[22] "clamoxyl"           "delacillin"         "dispermox"         
[25] "efpenix"            "flemoxin"           "hiconcil"          
[28] "histocillin"        "hydroxyampicillin"  "ibiamox"           
[31] "imacillin"          "lamoxy"             "metafarma capsules"
[34] "metifarma capsules" "moxacin"            "moxatag"           
[37] "ospamox"            "pamoxicillin"       "piramox"           
[40] "robamox"            "sawamox pm"         "tolodina"          
[43] "unicillin"          "utimox"             "vetramox"          

$loinc
[1] "16365-9" "25274-2" "3344-9"  "80133-2"

$ddd
$ddd$oral
$ddd$oral$amount
[1] 1.5

$ddd$oral$units
[1] "g"


$ddd$iv
$ddd$iv$amount
[1] 3

$ddd$iv$units
[1] "g"
\end{CodeOutput}
\end{CodeChunk}
%
In a data set containing antimicrobial names or codes (e.g.,~antimicrobial
prescription data), the \fct{as.ab} function can be used to transform all
values to valid antimicrobial codes.  Extra columns with the official name
and the defined daily dose (DDD) for intravenous administration could be
added using \fct{ab\_name} and \fct{ab\_ddd}.
%
\begin{CodeChunk}
\begin{CodeInput}
R> ab_example <- data.frame(agents = c("AMX", "Ceftriaxon", "Cipro"))
R> ab_example %>% mutate(
+    agents = as.ab(agents), agent_names = ab_name(agents),
+    ddd_iv = ab_ddd(agents, administration = "iv"))
\end{CodeInput}
\begin{CodeOutput}
  agents   agent_names ddd_iv
1    AMX   Amoxicillin    3.0
2    CRO   Ceftriaxone    2.0
3    CIP Ciprofloxacin    0.8
\end{CodeOutput}
\end{CodeChunk}
%
Coming back to the cleaning of the data, the columns for the antimicrobial
susceptibility test results (``AMX'', ``AMC'', ``CIP'', ``GEN'') need to be
checked to contain only standard values (``R'', ``S'', ``I'').
%
\begin{CodeChunk}
\begin{CodeInput}
R> data %>% 
+    select(AMX:GEN) %>% 
+    pivot_longer(everything(), names_to = "antimicrobials",
+    values_to = "interpretation") %>% 
+    count(interpretation)
\end{CodeInput}
\begin{CodeOutput}
# A tibble: 4 × 2
  interpretation     n
  <chr>          <int>
1 < 0.5 S          143
2 I               1105
3 R               4607
4 S               6145
\end{CodeOutput}
\end{CodeChunk}
%
The values contain some mixed values.  The \fct{as.rsi} function can be used
to clean these values and to assign a new class (\class{rsi}) for further
use of \pkg{AMR} functions.
%
\begin{CodeChunk}
\begin{CodeInput}
R> data <- data %>% 
+    mutate_at(vars(AMX:GEN), as.rsi)
R> data %>% 
+    select(AMX:GEN) %>% 
+    pivot_longer(everything(), names_to = "antimicrobials",
+    values_to = "interpretation") %>% 
+    count(interpretation)
\end{CodeInput}
\begin{CodeOutput}
# A tibble: 3 × 2
  interpretation     n
  <rsi>          <int>
1 S               6288
2 I               1105
3 R               4607
\end{CodeOutput}
\end{CodeChunk}
%
After this transformation, the \fct{eucast\_rules} function can be applied
to apply the latest resistance reporting guidelines.
%
\begin{CodeChunk}
\begin{CodeInput}
R> data <- data %>% eucast_rules()
\end{CodeInput}
\end{CodeChunk}
%
The output to the console lists the changes made to data:
%
\begin{CodeChunk}
\begin{CodeOutput}
The rules affected 508 out of 3,000 rows, making a total of 657 edits
=> added 0 test results

=> changed 657 test results
   - 11 test results changed from "S" to "I"
   - 473 test results changed from "S" to "R"
   - 85 test results changed from "I" to "R"
   - 19 test results changed from "I" to "S"
   - 33 test results changed from "R" to "I"
   - 36 test results changed from "R" to "S"
\end{CodeOutput}
\end{CodeChunk}
%
The data is now clean and ready for further analysis, for example, the
identification of multi-drug resistant microorganisms.  In this example, we
use the Dutch guideline to determine multi-drug resistance
\citep{Werkgroep_Infectiepreventie_WIP_undated-vu}).
%
\begin{CodeChunk}
\begin{CodeInput}
R> data <- data %>% mutate(mdro = mdro(., guideline = "nl"))
R> data %>% count(bacteria_name, mdro)
\end{CodeInput}
\begin{CodeOutput}
             bacteria_name     mdro    n
1         Escherichia coli Negative 1123
2         Escherichia coli Positive  395
3    Klebsiella pneumoniae Negative  237
4    Klebsiella pneumoniae Positive   89
5    Staphylococcus aureus Negative  730
6 Streptococcus pneumoniae Negative  426
\end{CodeOutput}
\end{CodeChunk}
%
According to the Dutch guideline, 484 multi-drug resistant microorganisms
were found in 3000 tested isolates.  No multi-drug resistance was found in
\emph{Staphylococcus aureus} and \emph{Streptococcus pneumoniae}.

As described in Section \ref{firstisolates}, the identification of first
isolates is essential for the reporting of resistance patterns.  Using the
\fct{filter\_first\_isolate} function and \fct{proportion\_df} in
combination with \fct{group\_by}, we get a complete resistance analysis per
hospital, bacteria, first isolate, and tested antimicrobial agent in one
call:
%
\begin{CodeChunk}
\begin{CodeInput}
R> resistance_proportion <- data %>% 
+    filter_first_isolate() %>%
+    group_by(hospital) %>% 
+    proportion_df()
R> head(resistance_proportion)
\end{CodeInput}
\begin{CodeOutput}
  hospital                  antibiotic interpretation     value
1        A                 Amoxicillin             SI 0.5796253
2        A                 Amoxicillin              R 0.4203747
3        A Amoxicillin/clavulanic acid             SI 0.8103044
4        A Amoxicillin/clavulanic acid              R 0.1896956
5        A               Ciprofloxacin             SI 0.7974239
6        A               Ciprofloxacin              R 0.2025761
\end{CodeOutput}
\end{CodeChunk}
%
From the console we get the information how many first isolates were
identified and used in the filter.

From here on, the data is ready for further analysis with functions for
plotting \citep[e.g.,~the \pkg{ggplot2} package][]{Wickham2009-ggplot2},
\pkg{AMR} extension functions for base \proglang{R} (e.g.,~\fct{summary}, \fct{plot}),
or \pkg{AMR} helper functions for plotting and basic modelling (e.g.,~\fct{ggplot\_rsi},
\fct{geom\_rsi}, \fct{resistance\_predict}).

%% -- Summary/conclusions/discussion -------------------------------------------

\section{Discussion} \label{sec:discussion}

For the first time, a free and open source software solution is available to
cover all aspects of working with antimicrobial resistance data.  The
\pkg{AMR} package provides functionalities that enable standardized and
reproducible workflows from raw laboratory data to publishable results, for
research and clinical workflows alike.  In the field of clinical
microbiology and infectious diseases, research and clinical workflows are
closely linked.  For example, a performed research study on the prevalence
of antimicrobial-resistant bacteria can have direct implications on the
choice of antimicrobial agents for the treatment of patients.  The \pkg{AMR}
package was developed to be used in any research or clinical setting where
the data analysis on microorganisms, antimicrobial resistance, antimicrobial
agents is required.

Both, researchers and clinicians rely on the data from electronic laboratory
information systems (LIS) where laboratory test results are processed,
stored, and archived.  Although some commercial solutions exist to conduct
medical microbiological data analysis, these solutions are not comprehensive
enough to apply antimicrobial resistance analysis for any clinical or
research setting.  Costs of these tools are a further constraint in
resource-limited settings.  Moreover, researchers and clinicians that
require data from multiple LIS sources to be used in multi-center studies
experience major barriers which cannot be solved by available commercial
solutions.

Firstly, simple codes for microorganisms show substantial differences
between different LIS and presumably correct taxonomic names are often
misspelled or outdated.  We analyzed the taxonomic names of bacteria used in
reports from seven different public health institutions that perform
microbiological diagnostics in the Netherlands and compared them with an
official scientific up-to-date source for microbial taxonomy, the Catalogue
of Life \citep{CoL-zq}.  These institutions cover microbiological
diagnostics for hospitals and primary care for ~15\% of the total Dutch
population.  All institutions reported outdated taxonomic names with a
maximum lag ranging between 34 and 41 years.  Given that antimicrobial
resistance guidelines are strongly based on the microbial taxonomy (some
rules only apply to a specific genus, other rules apply to a specific
family), it is crucial that this information is correct and timely updated. 
All institutions admitted that there was no standard operating procedure to
maintain their taxonomic reference data.  Implementing and maintaining the
taxonomic data for these and other institutions has been challenging, since
no common machine-readable, reliable and up-to-date resource for the
microbial taxonomy was publicly available.  For reliable reference data
about antimicrobial agents, this also holds true.  The \pkg{AMR} package
provides machine-readable reference data files for the complete microbial
taxonomy and for more than 500 antimicrobial agents.  Using functions
starting with \code{mo_*} and \code{ab_*}, names of microorganisms and
antimicrobial agents can be translated between different LIS codes or other
forms of text codes for microorganisms and consequently allows to merge data
sets from different sites with little effort.

Secondly, antimicrobial resistance interpretation guidelines
\citep{Leclercq2013-ml, Clinical_and_Laboratory_Standards_Institute2014-fb}
and taxonomic definitions of microorganisms are under constant change and
are continually published in dedicated peer-reviewed journals.  This is
further complicated by differences between local, regional, and national
guidelines.  Yet, comparability and reproducibility across setting and time
are key in research and clinics.  The \pkg{AMR} package functions
\fct{eucast\_rules} (to apply guidelines to data), \fct{mdro} (to check for
multi-drug resistance according to guidelines), or \fct{first\_isolate} (to
determine first isolates according to guidelines) address the needs to
standardize comparability, and empower data analysts beyond the capabilities
of their local LIS.  The \pkg{AMR} package can be used as an extra layer of
data validation when retrieving raw data from a LIS.  Overall, the
functionality of the \pkg{AMR} package has the potential to improve data
validity in clinical settings, to ease multi-center study workflows, and to
foster research reporting practices.  The inherent global nature of
antimicrobial resistances requires researchers, clinicians, and policy
makers to reach beyond the borders of their local laboratory.  The \pkg{AMR}
package can build the bridge to link these sources and further encourages
open science principles through its open source approach.

The \pkg{AMR} package also has limitations.  It does not introduce novel
statistical tests or models, nor does it add additional analytical
approaches for AMR research.  The calculation of the proportion of
susceptibility for more than one antimicrobial agent simultaneously (see
Section~\ref{sec:analyzingcalc}) seems simple but is subject to unclear
reporting in clinical practice \citep{Schechner2013, Ma2017}.  The lack of
clearly defined algorithms can lead to the effect that co-resistance rates
for more than one antimicrobial agent are dropped altogether
\citep{Baur2017}.  The inclusion of isolates that are tested for some agents
(\code{only_all_tested = FALSE}) or only isolates tested for all agents
(\code{only_all_tested = TRUE}) can have an imminent clinical impact on
patient care, if one combination of antimicrobial agents is preferred over
another.  Therefore, the \pkg{AMR} package provides different algorithms to
standardize this crucial calculation.  Unfortunately, unambiguous
methodology for determining the right algorithm is lacking in scientific
literature.  An analysis on the algorithms used in the \pkg{AMR} package and
their clinical impact is in preparation.

Reliable information about antimicrobial resistance is vital for clinical
decision-making in infectious diseases, since the outcome of local
antimicrobial resistance analyses support medical professionals/clinicians
in the treatment choices for their patients.  Moreover, when this
information can be reliably stratified by, for example, year, hospital, and
type of patients, new information can lead to new insights for choosing the
best antimicrobial therapy for patients suffering from infections.  The
\pkg{AMR} package enables this by providing all required analysis tools and
can therefore empower decision-making in infectious management.  The
\pkg{AMR} package is already being applied to this end in six hospitals in
the Netherlands.  The choice of empirical antimicrobial treatment (meaning;
choosing the initial therapy at a time of not knowing the infection-causing
pathogen) for septic non-post-surgical patients has been altered in at least
one Dutch hospital, by analyzing antimicrobial resistance data with the
\pkg{AMR} package.  The clinical effect of this adjustment is being studied
at the moment.  To improve the quality of such analyses, planned future
developments comprise the implementations of an imputation algorithm
specifically for antimicrobial agents, and method guidance for applying
prediction modelling in a health care setting based on patient-specific
properties.

Since the first package release, users from different public and private
settings have been suggesting additional functionalities, in particular, the
incorporation of country- or time-specific guidelines (e.g.,~\cite{Magiorakos2012}).
This community-centered development will be
continued and maintained by researchers at the University Medical Center
Groningen and data scientists at Certe Medical Diagnostics and Advice, both
non-profit public health organizations located in Groningen, the
Netherlands.  Moreover, a group of contributors from five different Dutch
health care institutions has been formed at the Dutch Association for
Medical Microbiology (Nederlandse Vereniging voor Medische Microbiologie -
NVMM) that also peer-review major changes to the package, including the
implementation of guideline updates.  This way, updates required for
scientific developments as well as maintaining consistent reproducibility
are ensured.  Updates to databases and guidelines included in the \pkg{AMR}
package are incorporated on a regular and automated basis, while preserving
version control.  Any function making use of guidelines (e.g.,~\fct{eucast\_rules})
refers to the latest implemented version of the
guideline by default.

The aim of the \pkg{AMR} package is to provide a comprehensive toolbox of
solutions for antimicrobial resistance data processing and analysis on an
institution- and country-independent scale for clinical practice and
research that are required according to international standards, but were
not available to date.


\section{Summary} \label{sec:summary}

This paper demonstrates the \pkg{AMR} package and its use for working with
antimicrobial resistance data.  It can be used to clean, enhance, and
analyze such data according to (inter)national recommendations and
guidelines while incorporating scientifically reliable reference data on
microbiological laboratory test results, antimicrobial agents, and the
biological taxonomy of microorganisms.  Consequently, it allows for
reproducible analyses, regardless of the many possible ways in which raw and
uncleaned data are stored in laboratory information systems.

While the burden of antimicrobial resistance is increasing worldwide,
reliable data and data analyses are needed to better understand current and
future developments.  Open source approaches, such as the \pkg{AMR} package
for \proglang{R}, have the potential to help democratizing the required
tools in the field for researchers, clinicians, and policy makers alike.  In
organizations or countries with very limited resources, this free and
open-source package could also overcome a financial limitation that would
otherwise hinder antimicrobial resistance analysis in these settings. 
Across settings, we believe the \pkg{AMR} package can be used to support
clinical decision-making in infection management by providing improved
insight into current local and regional resistance levels.  Furthermore,
data analysis approaches based on individual patient or microbiological
data, which the \pkg{AMR} package enables, fosters empowerment of laboratory
staff, infection control practitioners, and public health services.


%% -- Optional special unnumbered sections -------------------------------------

\section*{Computational details}

The results in this paper were obtained using \proglang{R}~4.0.2 in
\proglang{R}Studio~1.3.1093 \citep{RStudio} with the \pkg{AMR} package~1.5.0,
running under macOS Catalina~10.15.

\proglang{R} itself and all packages used are available from the
CRAN at \url{https://CRAN.R-project.org/}.  All development versions of the
\pkg{AMR} package are available at \url{https://github.com/msberends/AMR/}.

\section*{Acknowledgments}

The authors Matthijs S.  Berends and Christian F.  Luz contributed equally
to this publication.

For their contributions to the development of the \pkg{AMR} package, we
would like to thank (in alphabetical order) Judith M.  Fonville, Erwin E.A. 
Hassing, Eric H.L.C.M.  Hazenberg, Gwen Knight, Annick Lenglet, Bart C. 
Meijer, Sofia Ny, Rogier P.  Schade, Dennis Souverein, and Anthony
Underwood.

The development of the \pkg{AMR} package was partly supported by the
INTERREG V A (202085) funded project EurHealth-1Health
(\url{http://www.eurhealth1health.eu}), part of a Dutch-German cross-border
network supported by the European Commission, the Dutch Ministry of Health,
Welfare and Sport, the Ministry of Economy, Innovation, Digitalization and
Energy of the German Federal State of North Rhine-Westphalia and the
Ministry for National and European Affairs and Regional Development of Lower
Saxony.

Furthermore, the \pkg{AMR} package was developed as part of a project funded
by the European Commission Horizon 2020 Framework Marie Sk\l{}odowska-Curie
Actions (grant agreement number: 713660-PRONKJEWAIL-H2020-MSCA-COFUND-2015).

The funders had no role in study design, data collection and analysis,
decision to publish, or preparation of the manuscript.

%% -- Bibliography -------------------------------------------------------------
%% - References need to be provided in a .bib BibTeX database.
%% - All references should be made with \cite, \citet, \citep, \citealp etc.
%%   (and never hard-coded). See the FAQ for details.
%% - JSS-specific markup (\proglang, \pkg, \code) should be used in the .bib.
%% - Titles in the .bib should be in title case.
%% - DOIs should be included where available.

\bibliography{ref}

%% -- Appendix --------------------------------------------------------

\newpage

\begin{appendix}

\section{Included data sets} \label{app:datasets}

%
\begin{itemize}
  \item{\code{microorganisms}\\
  A \class{data.frame} containing 70,760 (sub)species with 16 columns
comprising their complete microbial taxonomy according to the Catalogue of
Life \citep{CoL-zq}.  Included microorganisms and their complete taxonomic
tree of all included (sub)species from kingdom to subspecies with year of
scientific publication and responsible author(s):
  %
  \itemize{
    \item{All 59,024 (sub)species from the kingdoms of Archaea, Bacteria,
    Chromista and Protozoa;}
    \item{All 9,582 (sub)species from these orders of the kingdom of Fungi:
    Eurotiales, Onygenales, Pneumocystales, Saccharomycetales,
    Schizosaccharomycetales and Tremellales;}
    \item{All 2,153 (sub)species from 47 other relevant genera from the
    kingdom of Animalia (like \emph{Strongyloides} and \emph{Taenia});}
    \item{All 14,338 previously accepted names of included (sub)species that
    have been taxonomically renamed.} }
%
  The kingdom of Fungi is a very large taxon with almost 300,000 different
(sub)species, of which most are not microbial (but rather macroscopic, such
as mushrooms).  Therefore, not all fungi fit the scope of the \pkg{AMR}
package.  By only including the aforementioned taxonomic orders, the most
relevant fungi are covered (such as all species of \emph{Aspergillus},
\emph{Candida}, \emph{Cryptococcus}, \emph{Histoplasma},
\emph{Pneumocystis}, \emph{Saccharomyces} and \emph{Trichophyton}).}
%
  \item{\code{antibiotics}\\
  A \class{data.frame} containing 464 antibiotic agents with 14 columns. 
All entries in this data set have three different identifiers: a human
readable EARS-Net code (as used by ECDC \citep{EARS_Net} and \pkg{WHONET}
\citep{WHO_Collaborating_Centre_for_Surveillance_of_Antimicrobial_Resistance_undated-vt}
and primarily used by this package), an ATC code (as used by the WHO
\citep{WHO_Collaborating_Centre_for_Drug_Statistics_Methodology2018-bj}),
and a CID code (Compound ID, as used by PubChem \citep{Kim2019-so}).  The
data set contains more than 5,000 official brand names from many different
countries, as found in PubChem.  Other properties in this data set are
derived from one or more of these codes, such as official names of
pharmacological and chemical subgroups, and defined daily doses (DDD).}
%
  \item{\code{antivirals}\\
  A \class{data.frame} containing 102 antiviral agents with 9 columns.  Like
the \code{antibiotics} data set, it contains ATC codes (as used by the WHO
\citep{WHO_Collaborating_Centre_for_Drug_Statistics_Methodology2018-bj}),
and a CID code (Compound ID, as used by PubChem \citep{Kim2019-so}), as well
as the official name and defined daily dose (DDD) for each antiviral
agent.}
%
  \item{\code{example_isolates}\\
  A \class{data.frame} containing test results of 2,000 microbial isolates. 
The data set reflects real patient data and can be used to practice AMR
analysis.  It is structured in the typical format of laboratory information
systems with one row per isolate and one column per tested antimicrobial
agent (i.e.,~an antibiogram).}
%  
  \item{\code{example_isolates_unclean}\\
  A \class{data.frame} containing test results of 3,000 microbial isolates
that require cleaning up before they can be used for analysis.  This data
set can be used to practice AMR analysis and is featured in Section~\ref{sec:reprex}.}
%  
  \item{\code{WHONET}\\
  A \class{data.frame} containing 500 observations and 53 columns, with the
exact same structure as an export file from \pkg{WHONET} 2019 software
\citep{WHO_Collaborating_Centre_for_Surveillance_of_Antimicrobial_Resistance_undated-vt}. 
Such files can be used with the \pkg{AMR} package, as this example data set
demonstrates.  The antibiotic test results are from the
\code{example_isolates} data set.  All patient names are created using
online surname generators and are only in place for practice purposes.}
 % 
\end{itemize}

\end{appendix}


%% -----------------------------------------------------------------------------

\end{document}
